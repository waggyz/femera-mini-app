
%%section{Femera quick start%label{sec:Quick-start}}

This quick start guide describes how to configure, build, and install
Femera and its dependencies with the recommended options.
\begin{verse}
\textbf{Note}:
System administrators and Femera developers should
review the Femera user and developer guide before continuing with this quick
start.
\end{verse}
Configure Femera before building and installing it.

\subsection{Quick configure\label{subsec:Quick-configure}}

Getting started with Femera begins with cloning the repository and
copying a configuration file to (\texttt{config.local}) as shown in
the example listing~\ref{lis:Quick-configure}.
\begin{verse}
\textbf{Best practice}:
Copy and edit an appropriate configuration file from
the (\texttt{examples/}) directory.
\end{verse}
\begin{comment}
tdd:tests/pre-build/config\_local\_is\_valid\_makefile.py
\end{comment}

\begin{lstlisting}[caption={Femera initial configuration steps and
time estimate},label={lis:Quick-configure},language=bash,float=ht]
#!/bin/bash
# Run these commands to download Femera source and begin configuring Femera.
#
git clone https://github.com/waggyz/femera-mini-app.git # 30-90 sc
cd femera-mini-app
cp examples/config.recommended config.local
\end{lstlisting}

Use a text editor to update the local configuration file (\texttt{config.local})
with your name, affiliation, email, and (recommended) a user-writable
installation directory as shown in listing~\ref{lis:config.recommended-head-9}.
Please review section~3 of the NASA Open Source Agreement (\texttt{NOSA-1-3.txt})
before editing this file.
\begin{comment}
tdd:tests/pre-build/file-does-exist.sh NOSA-1-3.txt
\end{comment}
\begin{verse}
\textbf{Best practice}:
Update the file (\texttt{config.local}) with
your name, affiliation, email, and desired installation directory.
\end{verse}
\begin{comment}
tdd:tests/pre-build/user\_did\_change\_BUILT\_BY.py
\end{comment}

Femera is intended to be used in a high-performance computing environment
with a network file system shared among compute nodes with different
capabilities.
Femera build configuration files use makefile syntax, as shown in
the example listing~\ref{lis:config.recommended-head-9}.
If the (\texttt{PREFIX}) entry is omitted or blank,
Femera and its dependencies will be installed in system directories.
\begin{verse}
\textbf{Best practice}:
Install Femera and its dependencies in a user directory and add this directory's
(\texttt{bin}, \texttt{lib}, \texttt{lib64})
sub-directories to the appropriate system paths.
\end{verse}
\begin{comment}
tdd:tests/post-install/femera\_is\_installed\_to\_local\_dir\_in\_path.py
\end{comment}

\begin{lstlisting}[caption={Excerpt of an example Femera configuration file
(\texttt{config.local})},
label={lis:config.recommended-head-9},language=make,float=ht]
#!/usr/bin/make
# config.local
#
# Please update the next line with your name, affiliation, and email address.
BUILT_BY:=Anonymous User (Example University)<anon@example.edu>
#
# Set PREFIX to the full path of a user-writable installation directory.
PREFIX:=/home/anon/local
#
# [...]
\end{lstlisting}

\begin{comment}
tdd:tests/pre-build/config\_recommended\_head\_n8\_is\_in\_guide.py
\end{comment}

Femera build scripts are written for the Linux Bash interpreter.
\begin{verse}
\textbf{Best practice}:
Set Bash as the default Linux shell when installing and using Femera.
\end{verse}

By default, Femera build scripts test that each step had the desired
result and report test results to the console.

\subsection{Quick build\label{subsec:Quick-build}}

Depending on how Femera is configured and what is already available
on the build system, Femera build scripts will typically download
and install several external dependencies before building Femera itself.
\begin{verse}
\textbf{Best practice}:
Build, test, tune, and install Femera on the same hardware it will be run on.
\end{verse}
\begin{comment}
tdd:tests/post-build/make\_test\_did\_check\_runtime\_fmrmodel.py
\end{comment}
\begin{verse}
\textbf{Note}:
The make targets (\texttt{test}) and (\texttt{tune})
should \textit{not} be built on a shared system (login) node.
\end{verse}
The time estimates in listings~\ref{lis:quick-start-user}
and~\ref{lis:quick-start-sys} are based
on building Femera and its dependencies in the recommended configuration,
and will vary with build system capabilities.
\begin{comment}
tdd:tests/pre-build/sys\_has\_fewer\_than\_5\_users.py
\end{comment}
\begin{verse}
\textbf{Note}:
These ``quick build'' terminal command examples
assume that the current working directory is the root of the
Femera source code repository (\texttt{femera-mini-app/}).
\end{verse}

\begin{center}
\begin{lstlisting}[caption={Build and install Femera in a user directory.},
label={lis:quick-start-user},language=bash,float=ht]
#!/bin/bash
# Build Femera from within the femera-mini-app/ directory.
#-------------------------------------
# Bash shell command       # Time Est.
#-------------------------------------
make external              # 30-90 mn
make mini                  #  5-10 mn
make test tune             #  1-2  hr
make install               #  1-2  mn
\end{lstlisting}
\begin{comment}
tdd:tests/post-build/make\_time\_in\_user\_guide\_is\_correct.py
\end{comment}
\end{center}

The make targets in
listings~\ref{lis:quick-start-user} and~\ref{lis:quick-start-sys},
summarized below, are described in more detail in
the Femera user and developer guide.

\begin{enumerate}
\item \texttt{make external}\\
Download, build, and install external libraries, tools, and applications
that are needed to build Femera.

\item \texttt{make mini}\\
Build the Femera~0.3 mini-app.
\item \texttt{make test tune}\\
Run Femera integration tests and optimize run-time parameters.
\item \texttt{make install}\\
Install Femera for research and production use.
\end{enumerate}

After installing Femera (\texttt{make install}),
commands like the examples in listing~\ref{lis:add-dir-to-path}
can be added to a Bash shell initialization script
(e.g.~\texttt{.bash\_profile}, \texttt{.bashrc})
using a text editor when Femera is installed to a local user directory.
\begin{comment}
tdd:tests/pre-build/sys\_bash\_is\_default.py
\end{comment}

\begin{lstlisting}[caption={Add user installation directories to
system search paths.},label={lis:add-dir-to-path},language=bash,float=ht]
#!/bin/bash
# Run commands like these in a bash console to update system paths temporarily,
# or append them to an initialization script (e.g. .bash_profile, .bashrc).

# Executable path
PATH=/home/anon/local/bin:$PATH
PATH=/home/anon/local/`fmrmodel`/bin:$PATH

# Library path
export LD_LIBRARY_PATH=/home/anon/local/lib64:$LD_LIBRARY_PATH
LD_LIBRARY_PATH=/home/anon/local/lib:$LD_LIBRARY_PATH
LD_LIBRARY_PATH=/home/anon/local/`fmrmodel`/lib:$LD_LIBRARY_PATH
\end{lstlisting}

Although installing Femera and its dependencies to a
user directory is recommended, Femera may also be installed to system
directories as usual for Linux applications.
\begin{verse}
\textbf{Note}:
Elevated privileges may be required to install Femera in system directories.
\begin{comment}
tdd:tests/pre-build/sys-user-has-install-privileges.sh
\end{comment}
\end{verse}
If there is no line in the file (\texttt{config.local})
to specify the installation directory (\texttt{PREFIX}), or it is blank,
Femera will be installed in system directories.
Use (\texttt{sudo}) to make the targets (\texttt{external} and
\texttt{install}) with elevated
privileges as shown in listing~\ref{lis:quick-start-sys}.
\begin{verse}
\textbf{Note}:
Un-installing Femera will \textit{not} remove dependecies installed in
system directories.
Dependencies installed in system directories must be removed individually.
\end{verse}

\begin{center}
\begin{lstlisting}[caption={Build and install Femera in system directories.},
label={lis:quick-start-sys},language=bash,float=ht]
#!/bin/bash
# Build Femera from within the femera-mini-app/ directory.
#-------------------------------------
# Bash shell command       # Time Est.
#-------------------------------------
sudo make external         # 30-90 mn
make mini                  #  5-10 mn
make test tune             #  1-2  hr
sudo make install          #  1-2  mn
\end{lstlisting}
\begin{comment}
tdd:tests/pre-build/quick\_start\_user\_sys\_listings\_is\_same\_with\_sudo\_lines.py
\end{comment}
\end{center}

Some common causes of build failures are addressed as frequently asked
questions~(FAQs) in the next section.

\subsection{Quick build frequently-asked questions~(FAQs)\label{subsec:Quick-faqs}}

The full list of FAQs for this version of Femera can be found
in~(\texttt{docs/FAQ.md}).
\begin{comment} or at
(\texttt{https://github.com/waggyz/femera-mini-app/docs/FAQ.md}).
\end{comment}
\begin{comment}
tdd:tests/pre-build/quick\_start\_faqs\_are\_in\_docs\_FAQ\_file.py
\end{comment}

\begin{verse}
\textbf{FAQ}:
\textit{Why does Make fail with permission errors when
building Femera installation targets?}\\
Installation to system directors often requires elevated
privileges. Prefix the installation (\texttt{make}) commands with (\texttt{sudo})
as shown in listing~\ref{lis:quick-start-sys}.

\textbf{FAQ}:
\textit{Why does Make fail with a bad file descriptor
(bad fd) error when building Femera?}\\
Most likely, this is because Bash is not the default
shell. Either configure the environment to use Bash by default, or
prefix all the (\texttt{make}) commands in listing~\ref{lis:quick-start-user}
with \texttt{(bash}) as shown in the example command below.\\
\texttt{bash make mini}

\textbf{FAQ}:
\textit{Why does building Femera or external dependencies fail
with a segmentation fault (segfault) or out of memory error?}\\
Compiling can use a very large amount of memory. Limit the number of
simultaneous threads for compiling by setting
(\texttt{MAKEJOBS}) in (\texttt{config.local})
as shown in the example below.\\
\texttt{MAKEJOBS:=4}

\end{verse}
After Femera has been built, tested, tuned, and installed,
there are several ways to get started using Femera.
This quick start guide has brief introductions to
Femera command-line use (section~\ref{subsec:cli-examples}),
interactive use (section~\ref{subsec:Intro-python-interactive}),
Python scripts (section~\ref{subsec:Intro-python-script}), and
C++ programming (section~\ref{subsec:Intro-cpp}).

\begin{verse}
\textbf{FAQ}:
\textit{Where are Femera examples and tutorials
installed?}\\
The make target (\texttt{install}) in listings~\ref{lis:quick-start-user}
and~\ref{lis:quick-start-sys} installs Femera documentation, tests, examples,
and tutorials in a sub-directory (\texttt{share/doc/femera/}) of the
installation directory described in section~\ref{subsec:Quick-configure}.
If a user directory is not provided, this sub-directory is installed
in a Linux system directory (e.g., \texttt{/usr/local/share/doc/femera/}).
\end{verse}
\begin{comment}
tdd:tests/post-install/docs-are-in-share-doc-femera-dir.sh
tdd:tests/post-install/tests-are-in-share-doc-femera-test-dir.sh
tdd:tests/post-install/examples-are-in-share-doc-femera-examples-dir.sh
tdd:tests/post-install/tutorial-is-in-share-doc-femera-tutorial-dir.sh
\end{comment}

More extensive example and tutorial descriptions can be found in the
Femera user and developer guide.

\subsection{Quick introduction to using Femera\label{subsec:quick-intro}}

placeholder

\subsubsection{Selected Femera command line examples\label{subsec:cli-examples}}

placeholder

\subsubsection{Introduction to interactive Femera use
\label{subsec:Intro-python-interactive}}

placeholder

\subsubsection{Introduction to Femera Python scripts
\label{subsec:Intro-python-script}}

placeholder

\subsubsection{Introduction to the Femera C++ library
\label{subsec:Intro-cpp}}

placeholder

\subsection{Quick cluster job submission introduction
\label{subsec:Intro-cluster-job}}

placeholder
.
